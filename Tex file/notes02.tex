\documentclass[12pt]{article}
\usepackage[english]{babel}
\usepackage{natbib}
\usepackage{url}
\usepackage[utf8x]{inputenc}
\usepackage{amsmath}
\usepackage{graphicx}
\graphicspath{{images/}}
\usepackage{parskip}
\usepackage{fancyhdr}
\usepackage{vmargin}
\usepackage{ctex}
\usepackage[noend]{algpseudocode}
\usepackage{algorithmicx,algorithm}
\usepackage{pythonhighlight}
\setlength{\parindent}{2em}
\setmarginsrb{3 cm}{2.5 cm}{3 cm}{2.5 cm}{1 cm}{1.5 cm}{1 cm}{1.5 cm}

\title{ 前沿数学专题讨论 PR02}								% Title
\author{22-23秋冬}								% Author
\date{19 Oct 2022}	% Date

\makeatletter
\let\thetitle\@title
\let\theauthor\@author
\let\thedate\@date
\makeatother

\pagestyle{fancy}
\fancyhf{}
\rhead{\theauthor}
\lhead{\thetitle}
\cfoot{\thepage}

\begin{document}

%%%%%%%%%%%%%%%%%%%%%%%%%%%%%%%%%%%%%%%%%%%%%%%%%%%%%%%%%%%%%%%%%%%%%%%%%%%%%%%%%%%%%%%%%

\begin{titlepage}
	\centering
    \vspace*{0.5 cm}
    \includegraphics[scale = 0.5]{logo.jpeg}\\[1.0 cm]	% University Logo
	% University Name
	\textsc{\Large \textbf{浙江大学数学科学学院} }\\[0.5 cm]				% Course Code
	\rule{\linewidth}{0.2 mm} \\[0.4 cm]
	{ \huge \bfseries \thetitle}\\
	\rule{\linewidth}{0.2 mm} \\[1.5 cm]
	{  \bfseries \thedate}\\
	\vskip 0.6 in
	\begin{minipage}{0.4\textwidth}
		\begin{flushleft} \large
			\emph{Submitted To:}\\
			张南松老师\\
                22-23秋冬学期\\
			\end{flushleft}
			\end{minipage}~
			\begin{minipage}{0.4\textwidth}
            
			\begin{flushright} \large
			\emph{Submitted By :} \\
			吴凡\\
            农业工程+统计学\\
		\end{flushright}
        
	\end{minipage}\\[2 cm]		    
	
\end{titlepage}

\tableofcontents
\pagebreak


\section{概述}
\subsection{简介}
SVM,英文全称为 Support Vector Machine,
中文名为支持向量机,由数学家Vapnik等人早在1963年提出。
在深度学习兴起之前,SVM一度风光无限,
是机器学习近几十年来最为经典的,
也是最受欢迎的分类方法之一。

\subsection{为何叫支持向量机}
SVM的本质模型特征空间中最大化间隔的\textbf{线性分类器},是一种\textbf{二分类模型}。

支持向量(Support vector):离分类超平面(Hyper plane)最近的样本点。

其核心的理念是:支持向量样本会对识别的问题起关键作用
\begin{figure}[htbp]
	\centering
	\includegraphics[width=8cm]{002.png}
\end{figure}

SVM的学习策略就是间隔最大化。

\subsection{直观理解}
请看下图,图中有分别属于两类的一些二维数据点和三条直线。如果三条直线分别代表三个分类器的话,
请问哪一个分类器比较好?
\begin{figure}[htbp]
	\centering
	\includegraphics[width=8cm]{003.png}
\end{figure}

我们凭直观感受应该觉得答案是H3。
首先H1不能把类别分开,
这个分类器肯定是不行的;H2可以,
但分割线与最近的数据点只有很小的间隔,
如果测试数据有一些噪声的话可能就会被H2错误分类
(即对噪声敏感、泛化能力弱)。H3以较大间隔将它们分开,
这样就能容忍测试数据的一些噪声而正确分类,
是一个泛化能力不错的分类器。

对于支持向量机来说,数据点若是p维向量,
我们用p-1维的超平面来分开这些点。
但是可能有许多超平面可以把数据分类。
最佳超平面的一个合理选择就是以最大间隔把两个类分开的超平面。
因此,SVM选择能够使离超平面最近的数据点的到超平面距离最大的超平面。

以上介绍的SVM只能解决线性可分的问题,为了解决更加复杂的问题,支持向量机学习方法有一些由简至繁的模型:
\begin{itemize}
	\item \textbf{线性可分SVM}:当训练数据线性可分时,通过硬间隔(hard margin,什么是硬、软间隔下面会讲)最大化可以学习得到一个线性分类器,即硬间隔SVM,如上图的的H3。
	\item \textbf{线性SVM}:当训练数据不能线性可分但是可以近似线性可分时,通过软间隔(soft margin)最大化也可以学习到一个线性分类器,即软间隔SVM。
	\item \textbf{非线性SVM}:当训练数据线性不可分时,通过使用核技巧(kernel trick)和软间隔最大化,可以学习到一个非线性SVM。
\end{itemize}

\section{线性可分SVM——hard margin}‘

考虑如下形式的线性可分的训练数据集:
$(X_1,y_1),(X_2,y_2),\cdots ,(X_N,y_N)$

% 其中$X_i\in \mathbb{R}^d$,是一个含有p个元素的列向量

$y_i\in \left\{+1,-1\right\} $,$y_i=+1$时表示$X_i$属于正类别,
$y_i=-1$时表示$X_i$属于负类别。

\subsection{超平面与间隔}

超平面方程为$W^TX+b=0$,可以规定法向量指向的一侧为正类,另一侧为负类。
下图画出了三个平行的超平面,法方向取左上方向。
\begin{figure}[htbp]
	\centering
	\includegraphics[width=10cm]{003.jpg}
\end{figure}

为了找到最大间隔超平面,
我们可以先选择分离两类数据的两个平行超平面,
使得它们之间的距离尽可能大。
在这两个超平面范围内的区域称为“间隔(margin)”,
最大间隔超平面是位于它们正中间的超平面。
这个过程如上图所示。

\subsection{数学模型}
判别模型:$y_i=sign(W^TX_i+b)$

模型判别正确$\Leftrightarrow \left\{
	\begin{array}{lr}
	W^TX_i+b>0,y_i=+1 &  \\
	W^TX_i+b<0,y_i=-1 &  
	\end{array}
	\right.\Leftrightarrow y_i(W^TX_i+b)>0,i=1,2,\ldots ,N$

margin(w,b)=$min_{w,b,X_i}$distance(w,b,$X_i$)=
$min_{w,b,X_i}\frac{\left\lvert w^TX_i+b\right\rvert }{\left\lVert w\right\rVert }$

SVM的中心思想:找到最大间隔超平面,即使margin取到最大值。则
优化问题可用如下数学模型表示:

\begin{align}
	\left\{
	\begin{array}{lr}
	max_{w,b}\frac{1}{\left\lVert w\right\rVert } &  \\
	s.t. y_i(w^TX_i+b)\geq 1,i=1,2,\ldots ,N &  
	\end{array}
	\right.
\end{align}

\begin{align}
	\Leftrightarrow \left\{
	\begin{array}{lr}
	min_{w,b} \frac{1}{2}w^Tw &  \\
	s.t. 1-y_i(w^TX_i+b)\leq  0,i=1,2,\ldots ,N &  
	\end{array}
	\right.
\end{align}

问题转化为纯粹的凸优化问题,用拉格朗日乘子法求最优解。

\begin{align}
	L(w,b,\lambda)=\frac{1}{2}w^tw+\sum_{i=1}^N\lambda_i\left[1-y_i(w^tX_i+b)\right] 
\end{align}
\begin{align}
	\Rightarrow \left\{
	\begin{array}{lr}
	min_{w,b} max_\lambda L(w,b,\lambda) &  \\
	s.t. \lambda_i\geq   0,i=1,2,\ldots ,N &  
	\end{array}
	\right.
\end{align}

引入本拉格朗日方程的对偶方程。
我们称上式所述问题为原始问题(primal problem), 
可以应用拉格朗日乘子法构造拉格朗日函数(Lagrange function)
再通过求解其对偶问题(dual problem)得到原始问题的最优解。
转换为对偶问题来求解的原因是:
\begin{enumerate}
	\item 对偶问题更易求解,由下文知对偶问题只需优化一个变量且约束条件更简单;
	\item 能更加自然地引入核函数,进而推广到非线性问题。
\end{enumerate}

对偶问题:
\begin{align}
	\left\{
	\begin{array}{lr}
	max_\lambda min_{w,b}  L(w,b,\lambda) &  \\
	s.t. \lambda_i\geq   0,i=1,2,\ldots ,N &  
	\end{array}
	\right.
\end{align}
\begin{align}
	\frac{\partial L}{\partial b}
	=\frac{\partial \frac{1}{2}w^tw+\sum_{i=1}^N\lambda_i\left[1-y_i(w^tX_i+b)\right]}{\partial b}
    =-\sum_{i=1}^N \lambda_iy_i=0
\end{align}
\begin{align}
	\Rightarrow \sum_{i=1}^N \lambda_iy_i=0
\end{align}

带入原式$L(w,b,\lambda)$中,则
\begin{align}
	L(w,b,\lambda)=\frac{1}{2}w^tw+\sum_{i=1}^N\lambda_i-\sum_{i=1}^N\lambda_iy_iw^TX_i
\end{align}
\begin{align}
	\frac{\partial L}{\partial w}
	=\frac{1}{2}w-\sum_{i=1}^N\lambda_iy_iX_i=0
\end{align}
\begin{align}
	\Rightarrow w^\ast =\sum_{i=1}^N\lambda_iy_iX_i
\end{align}

将$w^\ast$带入原式$L(w,b,\lambda)$中,
\begin{align}
	\begin{array}{lr}
		L(w,b,\lambda)=-\frac{1}{2}(\sum_{i=1}^N\lambda_iy_iX_i)^T(\sum_{j=1}^N\lambda_jy_jX_j)+\sum_{i=1}^N\lambda_i &  \\
		=-\frac{1}{2}\sum_{i=1}^N\sum_{j=1}^N\lambda_i\lambda_jy_iy_jX_i^TX_j+\sum_{i=1}^N\lambda_i&  
		\end{array}
\end{align}

因而,可通过逐项求导,求出$\lambda_i$的值。

\newpage
\section{算法实例}
\subsection{鸢尾花分类问题背景}
一名植物学爱好者收集了鸢尾花的一些测量数据:
花瓣的长度和宽度以及花萼的长度和宽度。
他还有一些鸢尾花分类的测量数据,
这些花已经被植物学专家鉴定为属于versicolor 、 
setosa 或 virginica 三个品种之一。

数据集主要属性:
\begin{figure}[h]
	\centering
	\includegraphics[width=8cm]{intro.png}
\end{figure}

每条记录包含5项基本信息:花萼的长度、花萼的宽度、
花瓣的长度、花瓣的宽度以及鸢尾花的类别。

\subsection{SVM分类}
现采用Python sklearn中的SVM模块对鸢尾花数据集进行分类。

采用的SVM分类方法分别为:
\begin{itemize}
	\item SVC(线性核)
	\item Linears(线性核)
	\item SVC(RBF)
	\item SVC(3次多项式核)
\end{itemize}

程序分类可视化结果如下:

\begin{figure}[H]
	\centering
	\includegraphics[width=16cm]{005.png}
\end{figure}

\subsection{代码}
\begin{python}
import numpy as np
import matplotlib.pyplot as pl
from sklearn import svm,datasets
from pylab import mpl

mpl.rcParams['font.sans-serif']='SimHei' #画图正常显示中文
mpl.rcParams['axes.unicode_minus']=False #决绝保存图像是负号‘-’显示方块的问题

def make_meshgrid(x,y,h=.02):
    """准备用于绘图的网格点
    参数
    ————————
    x:x轴数据点
    y:y轴数据点
    h:间隔距离
    返回值
    ——————————
    xx,yy:ndarray
    """
    x_min,x_max=x.min()-1,x.max()+1
    y_min,y_max=y.min()-1,y.max()+1
    xx,yy=np.meshgrid(np.arange(x_min,x_max,h),
                     np.arange(y_min,y_max,h))
    return xx,yy


def plot_contours(ax, clf, xx, yy, **params):
    """绘制分类器边界
    参数
    ————————
    ax:matplotlib.axes对象
    xx:网格点
    yy:网格点
    params:控制绘图的其它字典
    """

    Z = clf.predict(np.c_[xx.ravel(), yy.ravel()])
    Z = Z.reshape(xx.shape)
    out = ax.contour(xx, yy, Z, **params)
    return out


# 载入鸢尾花数据集
iris = datasets.load_iris()
# 为了后面方便绘图,这里只使用二个特征
X = iris.data[:, :2]
y = iris.target

C = 1.0
# 备用的各种模型设置
models = (svm.SVC(kernel='linear', C=C),
          svm.LinearSVC(C=C),
          svm.SVC(kernel='rbf', gamma=0.7, C=C),
          svm.SVC(kernel='poly', degree=3, C=C))
# 训练模型
models = (clf.fit(X, y) for clf in models)
# 各模型标题
titles = (u'SVC(线性核)',
          u'LinearsSVC(线性核)',
          u'SVC(RBF)',
          u'SVC(3次多项式核)')

# 把整个图划分成2*2网格
fig, sub = pl.subplots(2, 2, figsize=(12, 8))
pl.subplots_adjust(wspace=0.2, hspace=0.2)

X0, X1 = X[:, 0], X[:, 1]
xx, yy = make_meshgrid(X0, X1)

for clf, title, ax in zip(models, titles, sub.flatten()):
    plot_contours(ax, clf, xx, yy, alpha=0.8)
    ax.scatter(X0, X1, c=y, cmap=pl.cm.coolwarm, s=20, edgecolors='k')
    ax.set_xlim(xx.min(), xx.max())
    ax.set_ylim(yy.min(), yy.max())
    ax.set_xlabel(u'长')
    ax.set_ylabel(u'宽')
    ax.set_xticks(())  # 不显示坐标
    ax.set_yticks(())  # 不显示坐标
    ax.set_title(title)

pl.show()

\end{python}


% \begin{figure}[htbp]
% 	\centering
% 	\includegraphics[width=10cm]{tree.png}
% \end{figure}

\bibliographystyle{plain}
\bibliography{biblist}

\end{document}

