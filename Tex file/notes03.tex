\documentclass[12pt]{article}
\usepackage[english]{babel}
\usepackage{natbib}
\usepackage{url}
\usepackage[utf8x]{inputenc}
\usepackage{amsmath}
\usepackage{graphicx}
\graphicspath{{images/}}
\usepackage{parskip}
\usepackage{fancyhdr}
\usepackage{vmargin}
\usepackage{ctex}
\usepackage[noend]{algpseudocode}
\usepackage{algorithmicx,algorithm}
\usepackage{pythonhighlight}
\setlength{\parindent}{2em}
\setmarginsrb{3 cm}{2.5 cm}{3 cm}{2.5 cm}{1 cm}{1.5 cm}{1 cm}{1.5 cm}

\title{ 前沿数学专题讨论 PR03}								% Title
\author{22-23秋冬}								% Author
\date{26 Oct 2022}	% Date

\makeatletter
\let\thetitle\@title
\let\theauthor\@author
\let\thedate\@date
\makeatother

\pagestyle{fancy}
\fancyhf{}
\rhead{\theauthor}
\lhead{\thetitle}
\cfoot{\thepage}

\begin{document}

%%%%%%%%%%%%%%%%%%%%%%%%%%%%%%%%%%%%%%%%%%%%%%%%%%%%%%%%%%%%%%%%%%%%%%%%%%%%%%%%%%%%%%%%%

\begin{titlepage}
	\centering
    \vspace*{0.5 cm}
    \includegraphics[scale = 0.5]{logo.jpeg}\\[1.0 cm]	% University Logo
	% University Name
	\textsc{\Large \textbf{浙江大学数学科学学院} }\\[0.5 cm]				% Course Code
	\rule{\linewidth}{0.2 mm} \\[0.4 cm]
	{ \huge \bfseries \thetitle}\\
	\rule{\linewidth}{0.2 mm} \\[1.5 cm]
	{  \bfseries \thedate}\\
	\vskip 0.6 in
	\begin{minipage}{0.4\textwidth}
		\begin{flushleft} \large
			\emph{Submitted To:}\\
			张南松老师\\
                22-23秋冬学期\\
			\end{flushleft}
			\end{minipage}~
			\begin{minipage}{0.4\textwidth}
            
			\begin{flushright} \large
			\emph{Submitted By :} \\
			吴凡\\
            农业工程+统计学\\
		\end{flushright}
        
	\end{minipage}\\[2 cm]		    
	
\end{titlepage}

\tableofcontents
\pagebreak


\section{概述}
\subsection{简介}
EM,英文全称为 Expectation-Maximization Algorithm,
中文名为最大期望算法,是在概率模型中寻找参数最大似然估计或者最大后验估计的算法,其中概率模型依赖于无法观测的隐性变量。

最大期望算法经过两个步骤交替进行计算:
\begin{enumerate}
	\item 第一步是计算期望(E),利用对隐藏变量的现有估计值,计算其最大似然估计值;
	\item 第二步是最大化(M),最大化在E步上求得的最大似然值来计算参数的值。M步上找到的参数估计值被用于下一个E步计算中,这个过程不断交替进行。
\end{enumerate}

\subsection{算法思想}
假定你是一五星级酒店的厨师,现在需要把锅里的菜平均分配到两个碟子里。如果只有一个碟子乘菜那就什么都不用说了,但问题是有2个碟子,正因为根本无法估计一个碟子里应该乘多少菜,所以无法一次性把菜完全平均分配。

解法:大厨先把锅里的菜一股脑倒进两个碟子里,然后看看哪个碟子里的菜多,就把这个碟子中的菜往另一个碟子中匀匀,之后重复多次匀匀的过程,直到两个碟子中菜的量大致一样。 上面的例子中,平均分配这个结果是“观测数据z”,为实现平均分配而给每个盘子分配多少菜是“待求参数θ”,分配菜的手感就是“概率分布”。

EM算法的思想:
\begin{enumerate}
	\item 给θ自主规定个初值(既然我不知道想实现“两个碟子平均分配锅里的菜”的话每个碟子需要有多少菜,那我就先估计个值);
	\item 根据给定观测数据和当前的参数θ,求未观测数据z的条件概率分布的期望(在上一步中,已经根据手感将菜倒进了两个碟子,然后这一步根据“两个碟子里都有菜”和“当前两个碟子都有多少菜”来判断自己倒菜的手感);
	\item 上一步中z已经求出来了,于是根据极大似然估计求最优的θ’(手感已经有了,那就根据手感判断下盘子里应该有多少菜,然后把菜匀匀);
	\item 因为第二步和第三步的结果可能不是最优的,所以重复第二步和第三步,直到收敛(重复多次匀匀的过程,直到两个碟子中菜的量大致一样)。
\end{enumerate}
而上面的第二步被称作E步(求期望),第三步被称作M步(求极大化),从而不断的E、M。

\begin{figure}[htbp]
	\centering
	\includegraphics[width=8cm]{0003.png}
\end{figure}

EM迭代优化的路径是直线式的,可以看到每一步都会向最优值前进一步,而且前进路线是平行于坐标轴的,因为每一步只优化一个变量。

这犹如在x-y坐标系中找一个曲线的极值,然而曲线函数不能直接求导,因此什么梯度下降方法就不适用了。但固定一个变量后,另外一个可以通过求导得到,因此可以使用坐标上升法,一次固定一个变量,对另外的求极值,最后逐步逼近极值。对应到EM上,E步:固定θ,优化Q;M步:固定Q,优化θ;交替将极值推向最大。


\subsection{EM和MLE}

\begin{figure}[H]
	\centering
	\includegraphics[width=12cm]{0001.png}
\end{figure}

\begin{figure}[H]
	\centering
	\includegraphics[width=12cm]{0002.png}
\end{figure}
\begin{itemize}
	\item MLE是在“模型已定,参数未知”的情况下根据给定观察序列(所有序列服从同一分布)估计模型参数的估计方法。模型参数的准确性,跟观察序列直接相关。
	\item EM 算法就是含有隐变量的MLE。
\end{itemize}


\section{数学模型}
假设现在有m个独立样本$x=(x_1,x_2,\cdots,x_m)$,模型参数为$\theta$
\begin{equation}\nonumber
	\theta_{MLE}=argmax_\theta log p(x|\theta)
\end{equation}

对于EM算法:
\begin{equation}\nonumber
	\theta^{(t+1)}=arg max _\theta E_{z|x,\theta^{(t)}}[log P(x,z|\theta)]
\end{equation}

EM算法的迭代过程为,$\theta_0,\theta^{(1)},\theta^{(1)},\cdots,\theta^{(n)}$至收敛

下证:
\begin{equation}\nonumber
	P(x|\theta^{(t)})\leq P(x|\theta^{(t+1)})
\end{equation}

证明:
\begin{gather}\nonumber
	P(x|\theta)=\frac{P(x,z|\theta)}{P(z|x,\theta)}\\
	\Rightarrow log P(x|\theta)=log P(x,z|\theta)-log P(z|x,\theta)\\
	\int_{Z} P(z|x,\theta^{(t)}) log P(x|\theta) \,dz=log P(x|\theta)\int_{Z} P(z|x,\theta^{(t)}) \,dz=log P(x|\theta)\\
    \int_{Z} P(z|x,\theta^{(t)}) [log P(x,z|\theta)-log P(z|x,\theta)] \,dz \\
	=\int_{Z} P(z|x,\theta^{(t)}) log P(x,z|\theta) \,dz-\int_{Z} P(z|x,\theta^{(t)}) log P(z|x,\theta) \,dz
\end{gather}

令$Q(\theta,\theta^{(t)})=\int_{Z} P(z|x,\theta^{(t)}) log P(x,z|\theta) \,dz$

令$H(\theta,\theta^{(t)})=\int_{Z} P(z|x,\theta^{(t)}) log P(z|x,\theta) \,dz$

则
\begin{gather}\nonumber
	log P(x|\theta)=Q(\theta,\theta^{(t)})-H(\theta,\theta^{(t)})\\
    log P(x|\theta^{(t)})=Q(\theta^{(t)},\theta^{(t)})-H(\theta^{(t)},\theta^{(t)})\\
	log P(x|\theta^{(t+1)})=Q(\theta^{(t+1)},\theta^{(t)})-H(\theta^{(t+1)},\theta^{(t)})
\end{gather}

要证:
\begin{equation}\nonumber
	P(x|\theta^{(t)})\leq P(x|\theta^{(t+1)})
\end{equation}

等价于证:
\begin{equation}\nonumber
	log P(x|\theta^{(t)})\leq log P(x|\theta^{(t+1)})
\end{equation}

等价于证:
\begin{gather}\nonumber
	Q(\theta^{(t)},\theta^{(t)})\leq Q(\theta^{(t+1)},\theta^{(t)})\\
	H(\theta^{(t)},\theta^{(t)})\geq H(\theta^{(t+1)},\theta^{(t)})
\end{gather}

\begin{gather}\nonumber
	\theta^{(t+1)}=arg max_\theta Q(\theta,\theta^{(t)})\\
	Q(\theta^{(t+1)},\theta^{(t)})\geq Q(\theta,\theta^{(t)})
\end{gather}

由$\theta$的任意性,知$Q(\theta^{(t+1)},\theta^{(t)})\geq Q(\theta^{(t)},\theta^{(t)})$

由Jenson不等式:$E[log X]\leq log E[X]$

\begin{align}\nonumber
	&H(\theta^{(t+1)},\theta^{(t)})-H(\theta^{(t)},\theta^{(t)})\\
	&=\int_{Z} P(z|x,\theta^{(t)}) log \frac{P(z|x,\theta^{(t+1)})}{P(z|x,\theta^{(t)})} \,dz\\
	&=E[log \frac{P(z|x,\theta^{(t+1)})}{P(z|x,\theta^{(t)})}]\\
	&\leq log E(\frac{P(z|x,\theta^{(t+1)})}{P(z|x,\theta^{(t)})})\\
	&=log \int_{Z} P(z|x,\theta^{(t)}) \frac{P(z|x,\theta^{(t+1)})}{P(z|x,\theta^{(t)})} \,dz\\
	&=log \int_{Z} P(z|x,\theta^{(t+1)})  \,dz\\
	&=log 1=0
\end{align}

\section{EM的应用}

\subsection{EM的优缺点}
\begin{enumerate}
	\item 优点:算法简单,稳定上升的步骤能非常可靠地找到“最优的收敛值”;
	\item 缺点
	\begin{enumerate}
		\item EM算法的收敛速度,非常依赖初始值的设置,设置不当,计算时的代价是相当大的
		\item 对于大规模数据和多维高斯分布,其总的迭代过程,计算量大,迭代速度易受影响
		\item EM算法中的M-Step依然是采用求导函数的方法,所以它找到的是极值点,即局部最优解,而不一定是全局最优解。
	\end{enumerate}
\end{enumerate}

\subsection{EM的应用}
\begin{itemize}
	\item K-means聚类
	\item GMM(Gaussian Mixture Model, 高斯混合模型)
	\item HMM(Hidden Markov Model, 隐马尔可夫模型)
\end{itemize}
% \begin{figure}[htbp]
% 	\centering
% 	\includegraphics[width=10cm]{tree.png}
% \end{figure}

\bibliographystyle{plain}
\bibliography{biblist}

\end{document}

