\documentclass[12pt]{article}
\usepackage[english]{babel}
\usepackage{natbib}
\usepackage{url}
\usepackage[utf8x]{inputenc}
\usepackage{amsmath}
\usepackage{graphicx}
\graphicspath{{images/}}
\usepackage{parskip}
\usepackage{fancyhdr}
\usepackage{vmargin}
\usepackage{ctex}
\usepackage[noend]{algpseudocode}
\usepackage{algorithmicx,algorithm}
\usepackage{pythonhighlight}
\setlength{\parindent}{2em}
\setmarginsrb{3 cm}{2.5 cm}{3 cm}{2.5 cm}{1 cm}{1.5 cm}{1 cm}{1.5 cm}
\title{ 前沿数学专题讨论 PR06}								% Title
\author{22-23秋冬}								% Author
\date{21 Dec 2022}	% Date

\makeatletter
\let\thetitle\@title
\let\theauthor\@author
\let\thedate\@date
\makeatother

\pagestyle{fancy}
\fancyhf{}
\rhead{\theauthor}
\lhead{\thetitle}
\cfoot{\thepage}

\begin{document}

%%%%%%%%%%%%%%%%%%%%%%%%%%%%%%%%%%%%%%%%%%%%%%%%%%%%%%%%%%%%%%%%%%%%%%%%%%%%%%%%%%%%%%%%%

\begin{titlepage}
	\centering
    \vspace*{0.5 cm}
    \includegraphics[scale = 0.5]{logo.jpeg}\\[1.0 cm]	% University Logo
	% University Name
	\textsc{\Large \textbf{浙江大学数学科学学院} }\\[0.5 cm]				% Course Code
	\rule{\linewidth}{0.2 mm} \\[0.4 cm]
	{ \huge \bfseries \thetitle}\\
	\rule{\linewidth}{0.2 mm} \\[1.5 cm]
	{  \bfseries \thedate}\\
	\vskip 0.6 in
	\begin{minipage}{0.4\textwidth}
		\begin{flushleft} \large
			\emph{Submitted To:}\\
			张南松老师\\
                22-23秋冬学期\\
			\end{flushleft}
			\end{minipage}~
			\begin{minipage}{0.4\textwidth}
            
			\begin{flushright} \large
			\emph{Submitted By :} \\
			吴凡\\
            农业工程+统计学\\
		\end{flushright}
        
	\end{minipage}\\[2 cm]		    
	
\end{titlepage}

\tableofcontents
\pagebreak




\section{前言}
今天介绍的主题为强化学习中的一种actor critic的提升方式Deep Deterministic Policy Gradient (DDPG)算法,
DDPG最大的优势就是能够在连续动作上更有效地学习。

概括来说,RL要解决的问题是:让agent学习在一个环境中的如何行为动作(act), 从而获得最大的奖励值总和(total reward)。
这个奖励值一般与agent定义的任务目标关联。
agent需要的主要学习内容:第一是行为策略(action policy), 第二是规划(planning)。
其中,行为策略的学习目标是最优策略, 也就是使用这样的策略,
可以让agent在特定环境中的行为获得最大的奖励值,从而实现其任务目标。

行为(action)可以简单分为:
\begin{itemize}
	\item 连续的:如赛 车游戏中的方向盘角度、油门、刹车控制信号,机器人的关节伺服电机控制信号。
	\item 离散的:如围棋、贪吃蛇游戏。 Alpha Go就是一个典型的离散行为agent。
\end{itemize}

DDPG是针对连续行为的策略学习方法。在RL领域,DDPG主要从:PG -> DPG -> DDPG 发展而来。

相关基本概念:
\begin{enumerate}
	\item $s_t$:在t时刻,agent观察到的环境状态,比如观察到的环境图像,agent在环境中的位置、速度、机器人关节角度等;
	\item $a_t$:在t时刻,agent选择的行为(action),通过环境执行后,环境状态由$s_t$转换为$s_{t+1}$
	\item $r(s_t,a_t)$函数:环境在状态$s_{t}$ 执行行为$a_t$后,返回的单步奖励值;
	\item $R_t$:是从当前状态直到将来某个状态,期间所有行为所获得奖励值的加权总和,即discounted future reward
	\item $\gamma$:discounted rate,通常取0.99.
	\begin{align}
		R_t=\sum_{i=t}^T\gamma^{i-t}r(s_i,a_i)
	\end{align}
	\begin{figure}[htbp]
		\centering
		\includegraphics[width=8cm]{01.png}
	\end{figure}
\end{enumerate}

\subsection{PG}
R.Sutton 在2000年提出的Policy Gradient 方法,是RL中,学习连续的行为控制策略
的经典方法,其提出的解决方案是:
通过一个概率分布函数 $ \pi_{\theta}(s_{t} | \theta^{\pi})$ 来表示每一步的最优策略,
在每一步根据该概率分布进行action采样,获得当前的最佳action取值;即:
\begin{align}
	a_t \sim \pi_{\theta}(s_{t} | \theta^{\pi})
\end{align}


生成action的过程,本质上是一个随机过程;最后学习到的策略,也是一个随机策略(stochastic policy).


\subsection{DPG}
Deepmind的D.Silver等在2014年提出DPG: Deterministic Policy Gradient,
 即确定性的行为策略,每一步的行为通过函数$\mu$直接获得确定的值:
 \begin{align}
	a_t=\mu(s_t|\theta^\mu)
 \end{align}

 这个函数$\mu$即最优行为策略,不再是一个需要采样的随机策略。

 为何需要确定性的策略?简单来说,PG方法有以下缺陷:
 \begin{enumerate}
	\item 即使通过PG学习得到了随机策略之后,在每一步行为时,我们还需要对得到的最优策略概率分布进行采样,才能获得action的具体值;而action通常是高维的向量,比如25维、50维,在高维的action空间的频繁采样,无疑是很耗费计算能力的;
	\item 在PG的学习过程中,每一步计算policy gradient都需要在整个action space进行积分:
	\begin{align}
		\bigtriangledown _\theta=\int_S\int_A\rho(s)\pi_\theta(a|s)Q^\pi(s,a)dads
	\end{align}
	这个积分我们一般通过Monte Carlo 采样来进行估算,需要在高维的action空间进行采样,耗费计算能力。
 \end{enumerate}


\subsection{DDPG}

Deepmind在2016年提出DDPG,全称是:Deep Deterministic Policy Gradient,
是将深度学习神经网络融合进DPG的策略学习方法。
相对于DPG的核心改进是: 采用卷积神经网络作为策略函数$\mu$和Q函数的模拟,
即策略网络和Q网络;然后使用深度学习的方法来训练上述神经网络。

Q函数的实现和训练方法,采用了Deepmind 2015年发表的DQN方法 ,
即 Alpha Go使用的Q函数方法。


\section{DDPG算法}
\subsection{算法背景}

DDPG算法它吸收了Actor-Critic让Policy gradient单步更新的精华,而且还吸收让计算机学会玩游戏的DQN的精华,合并成了一种新算法,叫做Deep Deterministic Policy Gradient. 那 DDPG 到底是什么样的算法呢, 我们就拆开来分析。我们将DDPG分成'Deep'和'Deterministic Policy Gradient',然后'Deterministic Policy Gradient'又能细分为'Deterministic'和'Policy Gradient',接下来我们就开始一个个分析。

Deep顾名思义,就是走向更深层次,我们在DQN中提到过,我们使用一个经验池和两套结构相同,但参数更新频率不同的神经网络能有效促进学习。那我们也把这种思想运用到DDPG中,使DDPG也具备这种优良形式。但是DDPG的神经网络形式却比DQN的要复杂一点。

\begin{figure}[htbp]
	\centering
	\includegraphics[width=8cm]{2.jpg}
\end{figure}

\begin{figure}[htbp]
	\centering
	\includegraphics[width=8cm]{03.jpg}
\end{figure}

Policy Gradient相比其他的强化学习方法,它能被用来在连续动作上进行动作的筛选。而且筛选的时候是根据所学习到的动作分布随机进行筛选,而Deterministic有点看不下去,Deterministic说:我说兄弟,你其实在做动作的时候没必要那么不确定,反正你最终都只是要输出一个动作值,干嘛要随机。所以Deterministic就改变了输出动作的过程,只在连续动作上输出一个动作值。
\begin{figure}[htbp]
	\centering
	\includegraphics[width=8cm]{04.jpg}
\end{figure}

现在我们来说说DDPG中所用到的神经网络(粗略)。它其实和我们之前提到的Actor-Critic形式差不多,也需要有基于策略Policy的神经网络和基于价值Value的神经网络。但是为了体现DQN的思想,每种神经网络我们都需要再细分成两个,Policy Gradient这边,我们有估计网络和现实网络,估计网络用来输出实时的动作,供actor在现实中实行。而现实网络则是用来更新价值网络系统的。所以我们再来看看价值系统这边,我们也有现实网络和估计网络,他们都在输出这个状态的价值,而输入端却有不同,状态估计网络则是拿着当时Actor施加的动作当做输入。在实际运用中,DDPG这种做法确实带来了更有效的学习过程。

\subsection{算法相关概念和定义}
\begin{enumerate}
	\item 确定性行为策略$\mu$:定义为一个函数,每一步的行为可以通过$s_t=\mu(s_t)$计算获得。
	\item 策略网络:用一个卷积神经网络对$\mu$函数进行模拟拟合,这个网络我们就叫做策略网络,其参数为 $\theta^\mu$;
	\item behavior policy$\beta$:在RL训练过程中,我们要兼顾两个e:exploration和exploit(也就是之前说过的探索和开发);exploration的目的是探索潜在的更优策略,所以训练过程中,我们为action的决策机制引入随机噪声:将action的决策从确定性的过程变为一个随机过程,再从这个随机过程中采样得到action,下达给环境执行,过程如下图所示:
	\begin{figure}[htbp]
		\centering
		\includegraphics[width=12cm]{05.jpg}
	\end{figure}
	上述这个策略叫做behavior策略,用$\beta$来表示,
	这时RL的训练方式叫做off-policy。这里与$\epsilon-greedy$的思路是类似的。DDPG中,使用Uhlenbeck-Ornstein随机过程(下面简称UO过程),作为引入的随机噪声:
	UO过程在时序上具备很好的相关性,可以使agent很好的探索具备动量属性的环境。
    \item Q函数:即action-value函数,定义在状态$s_t$下,
	采取动作$a_t$后,且如果持续执行策略$\mu$
    的情况下,所获得的$R_t$期望值,用Bellman等式来定义:
	\begin{align}
		Q^\mu(s_t,a_t)=E[r(s_t,a_t)+\gamma Q^\mu(s_{t+1},\mu(s_{t+1}))]
	\end{align}
	可以看到,Q函数的定义是一个递归表达,在实际情况中,我们不可能每一步都递归计算Q的值,可行的方案是通过一个函数对Bellman等式表达进行模拟。
	\item Q网络:DDPG中,我们用一个卷积神经网络对Q进行模拟,这个网络我们就叫做Q网络,其参数为$\theta^Q$,采用了DQN相同的方法。
    \item  如何衡量一个策略$\mu$的表现:用一个函数J来衡量,我们叫做performance objective,针对off-policy学习的场景.
    \item 训练的目标:最大化$J_\beta(\mu)$,同时最小化Q网络的Loss,$\mu=argmax_\mu J(\mu)$
    \item 最优Q网络定义:具备最小化的Q网络Loss;训练Q网络的过程,就是寻找Q网络参数 
的最优解的过程,我们使用SGD的方法。
\end{enumerate}

\subsection{DDPG实现框架和算法}
以往的实践证明,如果只使用单个Q神经网络的算法,学习过程很不稳定,因为Q网络的参数在频繁梯度更新的同时,又用于计算Q网络和策略网络的gradient。基于此,DDPG分别为策略网络、Q网络各创建两个神经网络拷贝,一个叫做online,一个叫做target:
\begin{figure}[htbp]
	\centering
	\includegraphics[width=7cm]{22.png}
\end{figure}

在训练完一个mini-batch的数据之后,通过SGA/SGD算法更新online网络的参数,然后通过soft update算法更新target网络的参数。soft update是一种running average的算法:
\begin{figure}[htbp]
	\centering
	\includegraphics[width=6cm]{eq.png}
\end{figure}
\begin{itemize}
	\item target网络参数变化小,用于在训练过程中计算online网络的gradient,比较稳定,训练易于收敛。
	\item 参数变化小,学习过程变慢。
\end{itemize}

\subsubsection{算法流程}
\begin{figure}[htbp]
	\centering
	\includegraphics[width=12cm]{33.jpg}
\end{figure}

\begin{enumerate}
	\item actor根据behavior策略选择一个$a_t$,下达给gym执行该$a_t$。
	behavior策略是一个根据当前online策略$\mu$和随机UO噪声生成的随机过程, 从这个随机 过程采样 获得$a_t$的值。
	\begin{align}
		a_t=\mu(s_t|\theta^\mu)+N_t
	\end{align}
	\item  gym执行$a_t$,返回reward $r_t$和新的状态$s_{t+1}$;
	\item actor将这个状态转换过程(transition): $(s_t,a_t,r_t,s_{t+1})$存入replay memory buffer 中,作为训练online网络的数据集。
	\item 从replay memory buffer R中,随机采样N个transition数据,
	作为online策略网络、 online Q网络的一个mini-batch训练数据。我们用$(s_t,a_t,r_t,s_{t+1})$
	表示mini-batch中的单个transition数据。
	\item 计算online Q网络的 gradient:
	\item update online Q:采用adam optimizer更新$\theta^Q$;
	\item 计算策略网络的policy gradient;
	\item update online策略网络:采用Adam optimizer更新$\theta^Q$;
	\item soft update target网络$\mu'$和Q'
\end{enumerate}

\section{应用}
得益于DDPG算法能够对连续性信息进行学习,因而其在机器人自主避障、农业机器人自主采摘等
方面有着广阔的应用前景。现以COMPAG中“Collision-free path planning for a guava-harvesting robot based on 
recurrent deep reinforcement learning”一文具体介绍DDPG算法的农业机器人领域的应用。

\begin{figure}[htbp]
	\centering
	\includegraphics[width=12cm]{ro.png}
	\caption{(a)机器人实图(b)机械手连路坐标系}
\end{figure}

\begin{enumerate}
	\item 学习系统输入信息:植株图像
	\begin{figure}[htbp]
		\centering
		\includegraphics[width=12cm]{ro2.png}
		\caption{(a)原始图像(b)图像分割结果(c)图像3D建模}
	\end{figure}
	这一步是为了让机器人“看见”树枝和果实并作为输出信息,机器人需学习得到避开障碍物、成功抓取果实的最优路径。
	\item 建立DDPG算法模型,伪代码如下:
	\begin{figure}[htbp]
		\centering
		\includegraphics[width=12cm]{al1.png}
	\end{figure}
	\begin{figure}[htbp]
		\centering
		\includegraphics[width=12cm]{al2.png}
	\end{figure}
	\item 训练结果
	\begin{figure}[htbp]
		\centering
		\includegraphics[width=10cm]{out1.png}
	\end{figure}
	\begin{figure}[htbp]
		\centering
		\includegraphics[width=10cm]{out2.png}
	\end{figure}
	\begin{figure}[htbp]
		\centering
		\includegraphics[width=12cm]{out4.png}
		\caption{机器人使用循环DDPG作为现场路径规划器: (a)水果和障碍物定位;(b)无碰撞路径规划;(c)不与障碍物碰撞地接近目标.}
	\end{figure}
\end{enumerate}



\bibliographystyle{plain}
\bibliography{biblist}

\end{document}

