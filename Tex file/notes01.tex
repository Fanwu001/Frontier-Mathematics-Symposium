\documentclass[12pt]{article}
\usepackage[english]{babel}
\usepackage{natbib}
\usepackage{url}
\usepackage[utf8x]{inputenc}
\usepackage{amsmath}
\usepackage{graphicx}
\graphicspath{{images/}}
\usepackage{parskip}
\usepackage{fancyhdr}
\usepackage{vmargin}
\usepackage{ctex}
\usepackage[noend]{algpseudocode}
\usepackage{algorithmicx,algorithm}
\usepackage{pythonhighlight}

\setlength{\parindent}{2em}
\setmarginsrb{3 cm}{2.5 cm}{3 cm}{2.5 cm}{1 cm}{1.5 cm}{1 cm}{1.5 cm}

\title{ 前沿数学专题讨论 PR01}								% Title
\author{22-23秋冬}								% Author
\date{28 Sept 2022}	% Date

\makeatletter
\let\thetitle\@title
\let\theauthor\@author
\let\thedate\@date
\makeatother

\pagestyle{fancy}
\fancyhf{}
\rhead{\theauthor}
\lhead{\thetitle}
\cfoot{\thepage}

\begin{document}

%%%%%%%%%%%%%%%%%%%%%%%%%%%%%%%%%%%%%%%%%%%%%%%%%%%%%%%%%%%%%%%%%%%%%%%%%%%%%%%%%%%%%%%%%

\begin{titlepage}
	\centering
    \vspace*{0.5 cm}
    \includegraphics[scale = 0.5]{logo.jpeg}\\[1.0 cm]	% University Logo
	% University Name
	\textsc{\Large \textbf{浙江大学数学科学学院} }\\[0.5 cm]				% Course Code
	\rule{\linewidth}{0.2 mm} \\[0.4 cm]
	{ \huge \bfseries \thetitle}\\
	\rule{\linewidth}{0.2 mm} \\[1.5 cm]
	{  \bfseries \thedate}\\
	\vskip 0.6 in
	\begin{minipage}{0.4\textwidth}
		\begin{flushleft} \large
			\emph{Submitted To:}\\
			张南松老师\\
                22-23秋冬学期\\
			\end{flushleft}
			\end{minipage}~
			\begin{minipage}{0.4\textwidth}
            
			\begin{flushright} \large
			\emph{Submitted By :} \\
			吴凡\\
            农业工程+统计学\\
		\end{flushright}
        
	\end{minipage}\\[2 cm]		    
	
\end{titlepage}

\tableofcontents
\pagebreak


\section{问题背景}
小明看过以下6部电影,请你根据这6部电影的特点来推断小明
是否喜欢看某部电影。\\
	\begin{figure}[H]
		\centering
		\includegraphics[width=12cm]{01.png}
	\end{figure}
为建立决策树模型,引入2部小明没看过的电影作为对照。\\
	\begin{figure}[H]
		\centering
		\includegraphics[width=12cm]{02.png}
	\end{figure}
决策树模型的任务:从给定的训练数据集学得一个模型对新示例
进行分类。可看作对“当前示例属于这类吗?”这个问题的"决策"
或"判定"过程。\\

这个决策的过程通过决策树来完成。决策过程中提出的每个判定问题
都是对某个属性的"测试"。
\begin{figure}[htbp]
	\centering
	\includegraphics[width=8cm]{03.png}
\end{figure}

决策树学习的目的:产生一棵泛化能力强,即处理未见示例能力
强的决策树。


\newpage

\section{基本算法}
决策树模型基本流程遵循简单且直观的"分而治之"策略。

算法流程如下:
\begin{figure}[H]
	\centering
	\includegraphics[width=14cm]{09.png}
\end{figure}

\section{ID3算法模型}
一般而言,随着划分过程不断进行,
我们希望决策树的分支结点所包含的样本尽可能属于同一类别,
即结点的“纯度”(purity)越来越高.

“信息熵”为度量样本集合纯度常用的一种指标,假定当前样本集合D
中第k类样本所占的比例为$p_k(k=1,2,\cdots,|y|)$,则D的信息熵
定义为
\begin{align}
	Ent(D)=-\sum_{k = 1}^{|y|}p_klog_2p_k  
\end{align}

Ent(D)的值越小,则D的纯度越高。

假定离散属性a有V个可能的取值$\left\{a^1,a^2,\cdots,a^v \right\}$,
若使用a来对样本集D进行划分,则会产生V个分支结点,其中第v
个分支结点包含了D中所有在属性a上取值为$a^v$的样本,记为$D^v$,
可计算出$Ent_a(D)$,
于是可计算出用属性a对样本集D进行划分所获得的"信息增益"
\begin{align}
	Gain(D,a)=Ent(D)-Ent_a(D)=Ent(D)-\sum_{v = 1}^{V}\frac{|D^v|}{D}Ent(D^v) 
\end{align}

一般而言,信息增益越大,则意味着使用属性a来进行划分所获得的
"纯度提升"越大,因此可选择属性$a_\ast =arg max_{a\in A} Gain(D,a)$

举例:
\begin{figure}[H]
	\centering
	\includegraphics[width=12cm]{04.png}
\end{figure}

$Ent(D)=-(\frac{6}{8}*ln\frac{6}{8}+\frac{2}{8}*ln\frac{2}{8})=0.8112$

$Ent_{类型}(D)=\frac{3}{8}*(-\frac{3}{3}*ln\frac{3}{3}-\frac{0}{3}*ln\frac{0}{3})
+\frac{2}{8}*(-\frac{1}{2}*ln\frac{1}{2}-\frac{1}{2}*ln\frac{1}{2})
+\frac{3}{8}*(-\frac{2}{3}*ln\frac{2}{3}-\frac{1}{3}*ln\frac{1}{3})=0.5944$

因此,$Gain(D,类型)=Ent(D)-Ent_{类型}(D)=0.2168$

同理可求得,$Ent_{产地}(D)=0.3444,Ent_{票房}(D)=0.6887$

$Gain(D,产地)=Ent(D)-Ent_{产地}(D)=0.4668$

$Gain(D,票房)=Ent(D)-Ent_{票房}(D)=0.1225$

因此,Gain(D,产地)>Gain(D,类型)>Gain(D,票房)

属性"产地"的信息增益最大,于是它被选为划分属性,随后,
决策树学习算法对每个分支结点做进一步划分。

\begin{figure}[htbp]
	\centering
	\includegraphics[width=12cm]{05.png}
\end{figure}

上述过程为决策树算法中最常用的算法模型——ID3算法模型。

\section{C4.5 算法模型}

按照信息增益准则,$Gain(D,a)=Ent(D)-\sum_{v = 1}^{V}\frac{|D^v|}{D}Ent(D^v)$,
对可取值数目较多的属性有所偏好,如在“小明观影”的例子中,
若把“序号”作为划分属性,
此时,$Ent_{类型}(D)=6*\frac{1}{8}*(-\frac{1}{1}*ln\frac{1}{1})
+2*\frac{1}{8}*(-\frac{0}{1}*ln\frac{0}{1})=0$,则信息增益非常大,
而,若模型以“序号”作为划分属性,显然,此时这些分支结点的纯度
已达最大,然而,这样的决策树显然不具有泛化能力,无法对新样本进行有效预测

C4.5 算法模型不直接使用信息增益,而是使用"增益率"(gain ratio)来选择最优
划分属性。

\begin{align}
	Gain_ratio(D,a)=\frac{Gain(D,a)}{IV(a)}
\end{align}
其中,$IV(a)=-\sum_{v = 1}^{V} \frac{|D^v|}{|D|}log_2 \frac{|D^v|}{|D|}  $为属性a
的"固有值",属性a的可能取值数目越多,则IV(a)的值通常会越大。

增益率准则对可取值数目较少的属性有所偏好,因此C4.5算法并不是
直接选择增益率最大的候选划分属性,而是使用了一个启发式:
先从候选划分属性中找出信息增益高于平均水平的属性,再从中选择
增益率最高的。

\section{CART 决策树模型}

CART 决策树模型采用“基尼系数”(Gini index)来选择划分属性

\begin{align}
	Gini(D)=1-\sum_{k = 1}^{|y|}p_k^2  
\end{align}

直观来说,Gini(D)反映了从数据集D中随机抽取两个样本,其类别标记
不一致的概率,Gini(D)越小,则数据集D的纯度越高。
\begin{figure}[htbp]
	\centering
	\includegraphics[width=12cm]{06.png}
\end{figure}

属性a的基尼指数定义为:

\begin{align}
	Gini_index(D,a)=\sum_{v=1}^{V}\frac{|D^v|}{|D|}Gini(D^v)
\end{align}

\begin{figure}[htbp]
	\centering
	\includegraphics[width=12cm]{07.png}
\end{figure}
\begin{figure}[htbp]
	\centering
	\includegraphics[width=12cm]{08.png}
\end{figure}

于是,选择使得划分后基尼指数最小的属性作为最优划分属性,
即$a_\ast =arg min Giniindex(D,a)$

\section{决策树的剪枝}
决策树生成算法递归地产生决策树,直到不能继续下去为止,
这样产生的树往往会出现过拟合现象。
产生过拟合的原因在于学习时过多地考虑如何提高对
训练数据的正确分类,从而构建出过于复杂的决策树。
解决这个问题的办法是对已生成的决策树进行简化。

在决策树学习中将已生成的树进行简化的过程称为剪枝。
具体是从已生成的树上裁掉一些子树或叶节点,并将其根节点
或父节点作为新的叶节点,从而简化分类树模型。


剪枝策略分为预剪枝(prepruning)和后剪枝(post-pruning)。

预剪枝:在决策树生成过程中,对每个节点在划分前先进行估计,若当前节点的划分不能带来决策树泛化性能提升,则停止划分并将当前节点标记为叶节点。

后剪枝:先从训练集中生成一棵完整的决策树,然后自底向上地对非叶节点进行考察,若将该节点对应的子树替换成叶节点能带来决策树泛化能力的提升,则将该子树替换成叶节点。


\newpage

\section{代码实例}
使用银行营销数据集,这些数据共包括16个特征,输出标签有2个(“是”、“否”),即客户是否已订阅定期存款。我们将采用其中的三个特征“年龄”、“年平均余额”以及“该月与该客户最后一次联系的日期”
对客户是否订阅定期存款来进行预测。

\textbf{属性信息}\\
\begin{itemize}
	\item age 年龄(数字)
	\item job 职务:职务类型(类别:“管理员”,“蓝领”,“企业家”,“女佣”,“管理”,“退休” ,“自雇”,“服务”,“学生”,“技术员”,“待业”,“未知”)
	\item marital 婚姻:婚姻状况(类别:“已婚”,“离婚”,“单身”,“未知”)
	\item education 教育(类别:“未知”, “中学”, “小学”, “高等教育”)
	\item default 违约:信用违约吗?(类别:“否”,”是’)
	\item balance 收支:年平均余额
	\item housing 住房:有住房贷款吗?(分类:“否”,“是”)
	\item loan 贷款:有个人贷款吗?(类别:“否”,“是”)
\end{itemize}
等等

\textbf{输出类别信息}:客户是否已定阅?
\begin{python}
import pandas as pd

bm = pd.read_csv('./bank/bank.txt',sep=';')


X = bm[['age','balance','day']].values[:200,:]
y = bm['y'].values[:200]

bm.head()
\end{python}

\begin{figure}[htbp]
	\centering
	\includegraphics[width=12cm]{10.png}
\end{figure}

将数据集划为70$\%$作为训练集,30$\%$划分为测试集。
\begin{python}
from sklearn.model_selection import train_test_split

X_train, X_test, y_train, y_test = train_test_split(
    X, y, test_size=0.3, random_state=1, stratify=y)
\end{python}

使用scikit-learn训练一个决策树

\begin{python}
from sklearn.tree import DecisionTreeClassifier

# 采用gini系数进行度量,树的最大深度设置为4
tree = DecisionTreeClassifier(criterion='gini', 
                              max_depth=4, 
                              random_state=1)
# 训练
tree.fit(X_train, y_train)	
\end{python}

DecisionTreeClassifier()函数参数:

\begin{python}
class sklearn.tree.DecisionTreeClassifier(criterion=’gini’, splitter=’best’, max_depth=None, min_samples_split=2, min_samples_leaf=1, min_weight_fraction_leaf=0.0, max_features=None, random_state=None, max_leaf_nodes=None, min_impurity_decrease=0.0, min_impurity_split=None, class_weight=None, presort=False)
\end{python}

criterion:默认gini,即CART算法。

max depth:决策树最大深度,默认值是'None'.一般数据比较少或者特征少的时候可以不用管这个值,如果模型样本数量多,特征也多时,推荐限制这个最大深度,具体取值取决于数据的分布。常用的可以取值10到100之间,常用来解决过拟合。

random state:是用来设置决策树分枝中随机模式的参数,在高维度时sklearn决策树的特征随机性会很明显,低维度的数据(比如鸢尾花数据集),随机性几乎不会显现。高维数据下我们设置random state并配合splitter参数可以让模型稳定下来,保证同一数据集下是决策树结果可以多次复现,便于模型参数优化。


计算模型准确率:

\begin{python}
from sklearn.metrics import accuracy_score
y_train_pre = tree.predict(X_train)
y_test_pre = tree.predict(X_test)
print("training accuracy: %.3f"%accuracy_score(y_train_pre, y_train))
print("testing accuracy: %.3f"%accuracy_score(y_test_pre, y_test))	
\end{python}
Output:\\
training accuracy: 0.907\\
testing accuracy: 0.817

使用Graphviz库进行对决策树进行可视化:

\begin{python}
from pydotplus import graph_from_dot_data
from sklearn.tree import export_graphviz

# 生成.dot文件
dot_data = export_graphviz(tree,
                           filled=True, 
                           rounded=True,
                           class_names=['yes', 
                                        'no',],
                           feature_names=['年龄', 
                                          '收支',
                                          '联系日期'],
                           out_file=None) 
# export_graphviz(tree, out_file='tree.dot',feature_names=['age','balance','duration'])
# 将.dot文件存为图像文件
graph = graph_from_dot_data(dot_data) 
# 将图像文件写入
graph.write_png('tree.png')
\end{python}

\begin{figure}[htbp]
	\centering
	\includegraphics[width=12cm]{tree.png}
\end{figure}




\bibliographystyle{plain}
\bibliography{biblist}

\end{document}

